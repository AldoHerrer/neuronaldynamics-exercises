\documentclass[a4paper,10pt]{Exercises}
\usepackage{wrapfig}
\usepackage{hyperref}
\increasetextheight{2cm}
\increasetextwidth{1cm}
\begin{document}
\What{Neuronal Dynamics}
\Who{Professor Wulfram Gerstner\\Laboratory of Computational Neuroscience}
\SeriesTitle{Adaptive Exponential Integrate-And-Fire Model}
%%%%%%%%%%%%%%%%%%%%%%%%%%%%%%%%%%%%%%%%%%%%%%%%%%%%%%%%%%%%%%%%%%%%%%%%%%%%%%%%%%%%%%%%%
%%%%%%%%%%%%%%%%%%%%%%%%%%%%%%%%%%%%%%%%%%%%%%%%%%%%%%%%%%%%%%%%%%%%%%%%%%%%%%%%%%%%%%%%%
%%%%%%%%									%%%%%%%%%
%%%%%%%%				EXERCISE 1				%%%%%%%%%
%%%%%%%%									%%%%%%%%%
%%%%%%%%%%%%%%%%%%%%%%%%%%%%%%%%%%%%%%%%%%%%%%%%%%%%%%%%%%%%%%%%%%%%%%%%%%%%%%%%%%%%%%%%%
%%%%%%%%%%%%%%%%%%%%%%%%%%%%%%%%%%%%%%%%%%%%%%%%%%%%%%%%%%%%%%%%%%%%%%%%%%%%%%%%%%%%%%%%%
\newcommand{\erf}{\textrm{erf}}

%%%%%%%%%%%%%%%%%%%%%%%%%%%%%%%%%%%%%%%%%%%%%%%%


The goal of these exercises is to acquire some familiarity with Adaptive Exponential  Integrate-And-Fire model.
Download AdEx.py from the book's \href{http://neuronaldynamics.epfl.ch/lectures.html}{webpage}.  AdEx.py is a python module containing a function:  AdEx\_model. With this function, you can simulate a step current with amplitude of I\_amp and specific neuron properties.  The specific formulas implemented are described on page 137 of the book.  Once you have started ipython -pylab in the directory containing expLIF.py, simply type:
\begin{verbatim}
>> import AdEx
\end{verbatim}
to port AdEx.py onto your current session. Then you can simulate a step current in an adaptive exponential integrate-and-fire model by typing the following commands:  
\begin{verbatim}
>> AdEx.AdEx_model()
\end{verbatim}
which should trigger a plot with two panels, one is membrane voltage as a function of time, and the other is $w$ vs. $vm$.  To have information on the arguments of the function, simply open AdEx.py in any text editor. Using appropriate values of arguments, specifically $a$, $b$, $\tau_w$, $I_{amp}$ you can generate many different firing patterns with this model, which was not possible using exponential integrate-and-fire model or other linear models that you have seen in previous exercises.


\Exercise[]

First, try to get some intuition on shape of nullclines by plotting  or simply sketching them on a piece of paper and answering the following questions.
\Question How nullclines change with respect to values of $a$ and $I_{amp}$?
\Question What is the interpretation of parameter $b$?
\Question How flow arrows change as $\tau_w$ gets bigger?

\Exercise[]
Can you predict what would be the firing pattern if $a$ is relatively small?
To do so, consider the following 2 conditions:\\
1- A large jump ($b$) and a large time scale ($\tau_w$).\\
2- A small jump ($b$) and a small time scale ($\tau_w$).\\
Now try to simulate the above conditions, to see if your predictions were true.
You can use the following parameters:
\begin{table}[h]
\small
\label{table1}
\begin{tabular}{|l|l|l|l|l|l|l|l|l|l|l|}
\hline
 sets &$C$ ($pF$) & $gL$ ($nS$) & $EL$ ($mV$)  & $VT$ ($mV$) & $DeltaT$ ($mV$) & $a$ ($nS$) & $tau\_w$ ($ms$) & $b$ ($pA$) & $Vr$ ($mV$) & $I\_amp$ ($pA$)  \\ \hline
 set 1 & 200 & 10 & -70 & -50 & 10 & 2 & 30 & 0 & 58 & 200   \\ \hline
 set 2 & 200 & 12 & -70 & -50 & 2 & 2 & 300 & 60 & 58 & 500   \\ \hline
\end{tabular}
\end{table}

C,gL,EL,VT,DeltaT,a,tauw,b,Vr,I\_amp=200.0*pF,10.0*nS,-70.0*mV,-50.0*mV,10.0*mV,2.0*nS,30.0*ms,0.0*pA,-58.0*mV,200.0*pA\\
C,gL,EL,VT,DeltaT,a,tauw,b,Vr,I\_amp=200.0*pF,12.0*nS,-70.0*mV,-50.0*mV,2.0*mV,2.0*nS,300.0*ms,60.0*pA,-58.0*mV,500.0*pA

\Question How the inter spike interval (ISI) change in the above produced firing patterns? (You can use spike times stored in Spike_Times array to compute ISI.) 

\Exercise[]

The goal of this exercise is to generate some other spiking patterns.\\
\Question Using the following parameters and an appropriate value for I\_amp which you should find, generate a delayed accelerating pattern. (In a delayed accelerating pattern, spiking starts with a delay, and ISI decreases over time.)\\
\begin{table}[h]
\centering
\label{table2}
\begin{tabular}{|l|l|l|l|l|l|l|l|l|l|}
\hline
\small
$C$ ($pF$) & $gL$ ($nS$) & $EL$ ($mV$)  & $VT$ ($mV$) & $DeltaT$ ($mV$) & $a$ ($nS$) & $tau\_w$ ($ms$) & $b$ ($pA$) & $Vr$ ($mV$) & $I\_amp$ ($pA$)  \\ \hline
  200 & 12 & -70 & -50 & 2 & -10 & 300 & 0 & -58 & ? \\ \hline
\end{tabular}
\end{table}
\Question Try to plot nullclines in the same subplot which represents $w$ vs. $v\_m$ and try to explain how that firing pattern is generated with respect to this plot.
\Question Try to produce some other patterns such as regular bursting, delayed regular bursting, transient spiking, irregular spiking,etc.\\
To get an idea about appropriate range of values you can either refer to page 140 of the book or the article by \href{http://link.springer.com/article/10.1007/s00422-008-0264-7}{Naud et al. (2008)}.

\References
Naud, R., Marcille, N., Clopath, C., & Gerstner, W. (2008). Firing patterns in the adaptive exponential integrate-and-fire model. Biological cybernetics, 99(4-5), 335-347.




%###################
\end{document}
%###################
