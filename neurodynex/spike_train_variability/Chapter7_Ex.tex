\documentclass[a4paper,10pt]{Exercises}
\usepackage{wrapfig}
\usepackage{hyperref}
\increasetextheight{2cm}
\increasetextwidth{1cm}
\begin{document}
\What{Neuronal Dynamics}
\Who{Professor Wulfram Gerstner\\Laboratory of Computational Neuroscience}
\SeriesTitle{Variability of spike trains and neural codes}
%%%%%%%%%%%%%%%%%%%%%%%%%%%%%%%%%%%%%%%%%%%%%%%%%%%%%%%%%%%%%%%%%%%%%%%%%%%%%%%%%%%%%%%%%
%%%%%%%%%%%%%%%%%%%%%%%%%%%%%%%%%%%%%%%%%%%%%%%%%%%%%%%%%%%%%%%%%%%%%%%%%%%%%%%%%%%%%%%%%
%%%%%%%%									%%%%%%%%%
%%%%%%%%				EXERCISE 7				%%%%%%%%%
%%%%%%%%									%%%%%%%%%
%%%%%%%%%%%%%%%%%%%%%%%%%%%%%%%%%%%%%%%%%%%%%%%%%%%%%%%%%%%%%%%%%%%%%%%%%%%%%%%%%%%%%%%%%
%%%%%%%%%%%%%%%%%%%%%%%%%%%%%%%%%%%%%%%%%%%%%%%%%%%%%%%%%%%%%%%%%%%%%%%%%%%%%%%%%%%%%%%%%
\newcommand{\erf}{\textrm{erf}}

%%%%%%%%%%%%%%%%%%%%%%%%%%%%%%%%%%%%%%%%%%%%%%%%

The goal of these exercises is to acquire some familiarity with \href{http://neuronaldynamics.epfl.ch/online/Ch7.S2.html}{variability of spike trains, and  in particular poisson model of spike generation}.
%more introduce the topic
Download Chapter7\_Ex.py  from  \href{http://neuronaldynamics.epfl.ch/lectures.html}{here}.  Chapter7\_Ex.py is a python module containing 2 main functions:  Binomial\_SampleGenerator , ExpDist\_SampleGenerator. The former, generates samples from a  binomial distribution and the latter, generates samples from and exponential distribution . Using those, you can generate spike trains with specific statistics and compare them. Once you have started ipython -pylab in the directory containing Chapter7\_Ex.py, simply type:
\begin{verbatim}
>> import Chapter7_Ex
\end{verbatim}
to port Chapter7\_Ex.py onto your current session. Then call the functions simply by typing:
\begin{verbatim}
>> Chapter7_Ex.Name_of_function (Arguments)
\end{verbatim}

\Exercise[]
The aim of this exercise is to first generate spike trains with 2 methods (part 1.1 and 1.2) and then compare the inter spike interval histogram of them to see if the 2 methods are equivalent according to simulation results (part 1.c). (The theoretical equivalence is shown during the lecture.)  
\Question Using forward\_sampling function, you can generate a spike train. The probability of firing in each time step with this function is $\rho*\Delta t$, and the spikes are generated by generating a random number in each time step and the decision of weather there is a spike in that time point or not, is based on the random value generated. \\
Observe the histogram of inter spike interval, and try to interpret it. Do you see the similarity to any of the well-known probability distributions?
\Question Using inverse\_transform\_sampling function, you can generate another spike train. This function generates samples from an exponential distribution to define the interval between to consecutive spikes. Observe the ISI histogram and interpret it.\\ (Inverse transform sampling is a basic method for generating sample numbers at random from any probability distribution given its cumulative distribution function.)
\Question Using plots function, you can plot ISI histograms driven from part 1 and 2, in addition to the exponential distribution on the same figure. How well do they match?Try o change the value of some arguments such as delta\_t and explain how they affect the resulting histogram.

\Exercise[]
The aim of this exercise is to do the same tasks as in the first part, but adding the refractoriness to our functions which  produce spike trains with the 2 aforementioned methods and probability distributions. Adding the refractoriness to the model, simply means the neuron will not fire for a certain period of time after it spikes at a specific point, and then it can fire according to the same probability distributions as in first part.
\Question Use forward\_sampling\_with\_refractoriness, inverse\_transform\_sampling\_with\_refractoriness, and plots\_with\_refractoriness functions to do the same tasks as in the first part but with refractoriness, and try to answer the same questions.

%###################
\end{document}
%###################
