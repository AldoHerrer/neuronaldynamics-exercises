\documentclass[12pt]{article}
\usepackage[a4paper]{geometry}

\usepackage[lastexercise]{exercise}

%%%%%%%%%%% Preamble for exercise sheets
\pagestyle{empty}
\usepackage{vmargin}
\setmarginsrb{2.5cm}{2.5cm}{2.5cm}{2.5cm}{0cm}{0cm}{0cm}{0cm}
\setlength{\parindent}{0cm}
\linespread{1.0}
\newcommand{\Title}[1]{\makebox[\textwidth][c]{{\large\scshape{#1}}}\\[5mm]}
\newcommand{\What}[1]{\makebox[\textwidth][c]{\large{#1}}\\[5mm]}
\newcommand{\Who}[1]{{\parbox[t]{\linewidth}{\centering\large{#1}}\\[1.0cm]}}
%%%%%%%%%%%

\begin{document}

\What{Neuronal Dynamics: Python Exercises}
\Who{Professor Wulfram Gerstner\\Laboratory of Computational Neuroscience, EPF Lausanne}
\Title{Leaky Integrate-And-Fire Model}

\begin{Exercise}[]

Use the function \emph{LIF.LIF\_Step} to simulate a Leaky Integrate-And-Fire neuron stimulated by a current step of a given amplitude. The goal of this exercise is to modify the provided python functions and use the \emph{numpy} and \emph{matplotlib} packages to answer the following questions.

\Question What is the minimum current step amplitude \emph{I\_amp} to elicit a spike with model parameters as given in \emph{LIF.LIF\_Step}?
\Question Plot the injected values of current step amplitude against the frequency of the spiking response (you can use the inter-spike interval to calculate this -- let the frequency be $0Hz$ if the model does not spike, or emits only a single spike) during a $500ms$ current step.

\end{Exercise}

\begin{Exercise}[]

Use the function \emph{LIF.LIF\_Sinus} to simulate a Leaky Integrate-And-Fire neuron stimulated by a sinusoidal current of a given frequency. The goal of this exercise is to modify the provided python functions and use the \emph{numpy} and \emph{matplotlib} packages to plot the	 amplitude and frequency gain and phase of the voltage oscillations as a function of the input current frequency.

\Question For input frequencies between $0.1Hz$ and $1.Hz$, plot the input frequency against the resulting \emph{amplitude of subthreshold oscillations}  of the membrane potential. If your neuron emits spikes at high stimulation frequencies, decrease the amplitude of the input current.

\Question For input frequencies between $0.1Hz$ and $1.Hz$, plot the input frequency against the resulting \emph{frequency and phase of subthreshold oscillations}   of the membrane potential. Again, keep your input amplitude in a regime, where the neuron does not fire action potentials.

\end{Exercise}

\end{document}