\documentclass[12pt]{article}
\usepackage[a4paper]{geometry}

\usepackage[lastexercise]{exercise}

%%%%%%%%%%% Preamble for exercise sheets
\pagestyle{empty}
\usepackage{vmargin}
\setmarginsrb{2.5cm}{2.5cm}{2.5cm}{2.5cm}{0cm}{0cm}{0cm}{0cm}
\setlength{\parindent}{0cm}
\setlength{\parskip}{2mm}
\linespread{1.0}
\newcommand{\Title}[1]{\makebox[\textwidth][c]{{\large\scshape{#1}}}\\[5mm]}
\newcommand{\What}[1]{\makebox[\textwidth][c]{\large{#1}}\\[5mm]}
\newcommand{\Who}[1]{{\parbox[t]{\linewidth}{\centering\large{#1}}\\[1.0cm]}}
%%%%%%%%%%%

\begin{document}

\What{Neuronal Dynamics: Python Exercises}
\Who{Professor Wulfram Gerstner\\Laboratory of Computational Neuroscience, EPF Lausanne}
\Title{Numerical simulation of Oja's hebbian learning rule}

\begin{Exercise}[]

Download \texttt{oja.py} from book's webpage. Create a script file \verb|answers.py| in the
same directory, and give it the following header:
\begin{verbatim}
from numpy import *
from scipy import *
from oja import *
\end{verbatim}
You will type the code for your answers right below. When you want to execute your code, 
open ipython (pylab) in a terminal and type: 
\begin{verbatim}
>> import answers
\end{verbatim}
The \texttt{oja} module contains two documented functions:
\begin{itemize}
\item \verb|make_cloud(...)| which generates a 2D elliptic gaussian cloud of datapoints
\item \verb|learn(...)| which runs Oja's learning rule on the data and returns the time course of the weight vector
\end{itemize}

In your answers you will need to plot data, which can be done using pylab.

\Question
Run the learning rule on a \emph{circular} data cloud (\verb|ratio=1|). Plot the time course
of both components of the weight vector. Do it many times. What happens?

\Question
Repeat the previous question with an \emph{elongated} elliptic data cloud (e.g. \verb|ratio=0.3|). 
Again, do it several times. What is the
difference with a circular data cloud, in terms of learning curve?
 
\Question
Try to change the orientation of the ellipsoid (try several different angles). What does Oja's rule do?
Hint: plot the learned weight vector in 2D space, and relate its orientation to that of the ellipsoid.

\Question
The above work assumes that the input activities can be negative (indeed the input were statistically centered).
In real neurons, if we think of the ``activity'' as the firing rate, this cannot be less than zero.
Try again the previous question, applying the learning rule on \verb|5 + make_cloud(...)|,
which centers the data around \verb|(5,5)|.
Draw your conclusions. Can you think of a modification of the rule?


\end{Exercise}

\end{document}